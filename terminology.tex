\chapter{Terminology}
Some terminology are used as different meanings by different people. To avoid confusion, we will now present how we use these terminology in this document.

\begin{itemize}
	\item \emph{PE} is short for Processing Element, which is the basic component of dispel4py graph (detailed later). In most cases, a \emph{PE} is equivalent to a \emph{task} in a workflow. We will use the term \emph{PE} instead of \emph{task} when talking about dispel4py.
	\item \emph{Task} means a specific computation job in a workflow. Each workflow consists of many \emph{tasks}. Sometimes, we use \emph{(sub-)task} instead of \emph{task} for better clearance, but they mean the same. It will be used when we are not talking specifically about dispel4py.
	\item \emph{Node} means a computer in a network, usually one executing some computational jobs. It will usually be considered as a single processor (core), so we will often consider each processor (core) as one \emph{node}. Usually, one \emph{node} carries one \emph{task}, so they are sometimes used exchangeably.
	\item \emph{Unit} means a piece of data produced by some \emph{task}s / \emph{PE}s. Usually, each \emph{task} / \emph{PE} will produce many \emph{unit}s of data with the same format.
	\item \emph{Data-streaming} means the data produced is not going to be cached on disk (or other permanent storage devices), but to directly send (\ie stream) to the corresponding receiver node(s).
	\item \emph{Pipeline} refers to the property that the following nodes start to process data when the first \emph{unit} of data is produced (\ie don't have to wait until all data has been produced).
\end{itemize}

