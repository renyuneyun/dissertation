\chapter{Terminology}
To avoid confusion, we present how we use terminology in this document.

\newenvironment{term_box}{\begin{tcolorbox}[enhanced,width=5in,size=fbox,
    fontupper=\large\bfseries,drop shadow southwest,sharp corners]}{
\end{tcolorbox}}

\begin{itemize}

\item \emph{PE} is short for Processing Element, which is the basic component of \dpy graph (detailed later). In most cases, a \emph{PE} is equivalent to a \emph{task} in a workflow. We will use the term \emph{PE} instead of \emph{task} when talking about \dpy.

\item \emph{Task} means a specific computation job in a workflow. Each workflow consists of many \emph{tasks}. Sometimes, we use \emph{(sub-)task} instead of \emph{task} for clarity, but they mean the same. It will be used when we are not talking specifically about \dpy.

\begin{term_box}
Although \emph{PE} and \emph{task} are usually interchangeable in our context, there is a difference in the common case. Normally, \emph{task} is used in the task-oriented environment so they will exit after processing; \emph{PE} is a term in \dpy (which is a \emph{data-streaming} system), so they will keep running and waiting for more data after processing a unit of data until no more data coming in.
\end{term_box}	
	
\item \emph{Node} means a computer in a network, usually one executing some computational jobs. Because modern computers usually consists of many \emph{core}s, it is not ideal to consider each of the computers (which may have different number of \emph{cores}) as a standalone entity. A better practice would be considering each \emph{node} as a single-core machine.

\item \emph{Core} refers to a CPU core. A modern computer consists of many \emph{core}s, so \emph{core} is not the same as \emph{node}. Usually we won't use \emph{core} directly because programming doesn't directly operate on \emph{core}s.

\begin{term_box}
Although \emph{node} and \emph{core} are different in definition, we would usually consider each \emph{node} as a single-\emph{core} machine, which means they are the same in our context (and we prefer to use \emph{node}).
\end{term_box}	

\begin{term_box}
Notice that each \emph{task} or \emph{PE} is usually programmed with no aware of multi-cores. However, they can wield multi-threading to perform necessary light-weight (i.e. not computational-intensive) tasks (such as communication).
\end{term_box}
	
\item \emph{Unit} means a piece of data produced by some \emph{task}s / \emph{PE}s. Usually, each \emph{task} / \emph{PE} will consume or produce many \emph{unit}s of data with the same format (on each data stream).

\item \emph{Data-streaming} means the data produced is not going to be cached on disk (or other permanent storage devices), but to be directly sent (\ie streamed) to the corresponding receiver node(s).

\item \emph{Pipeline} refers to the property that the following nodes start to process data when the first \emph{unit} of data is supplied (\ie it does not wait until all data has been produced by upstream \emph{PE}s).

\end{itemize}

