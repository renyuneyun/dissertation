\chapter{Conclusion and Future Work}
In this document, we presented the important role of scientific workflows for today's scientific research, the classification of workflow management systems (WMSs) and \dpy, a data-streaming WMS. They formed the the background to our work. Then, we discussed two extensions, \tincdep and \tdynexp, to \dpy, both give \dpy more dynamics and enable further extensions or optimization. We described our design and implementation. Finally, we presented our evaluation and measurement of our extension and showed that our extension gives \dpy more dynamics at a low cost and sometimes even with better performance.

Our extensions to \dpy, \tincdep and \tdynexp, jointly form a significant starting point of future extension or optimization to \dpy or other data-streaming WMSs. For example, one can utilize and extend the \tincdep mechanism to dynamically schedule PEs to the most appropriate nodes, or utilize and extend the \tdynexp mechanism to enable more complex runtime workflow self-management behaviours. Some options are described more detailed in the Future Work section in this chapter.

The time limit prevented us to explore measurement with a representative sets of workflows and investigate scalability. We are also aware of the limitation in our implementation. For example the single coordinator may become a bottleneck if communication increases; frequent operations on MPI communicators may also affect the performance. Possible extensions or optimizations is described in the rest of this chapter, so we or other developers can extend our work. Even though, we consider we have made good work during the MSc dissertation period.

\section{Future Work}
The rest part of this chapter will describe possible future work. They consist of two aspects:
\begin{enumerate}
	\item Possible optimizations to our current implementation;
	\item Possible extension built upon our work.
\end{enumerate}

If we were allowed more time, we would explore the possible optimizations first, and then the extension.

\subsection{Possible Optimization}
In our current implementation, there are some compromises (which have been briefly mentioned in the preceeding chapters) because of the time limit. Given enough time, we would explore which ways will be better and implement the best. These optimizations are considered by us:

\begin{enumerate}
	\item The \emph{process} and \emph{write} steps are combined together, and they are concurrent with the \emph{read} steps during processing. However, all they three can be executed in parallel in principle. Other aspects, such as listening to signals from the coordinator and performing actions, can also be executed in parallel in principle.
	\item When spawning more nodes, the current implementation always spawns a fixed pre-defined number of nodes. However, there may be a smart way to specify this number dynamically according to the workloads and known application behaviour.
	\item After spawning new nodes, the current implementation will connect all nodes together (by merging the \emph{intercommunicator}) in one \emph{intracommunicator} in order to make all worker nodes inter-communicable. However, in the MPI implementation, the merging operation takes time and during this time no data / message can be transmitted. We will call \lstinline|Comm.Dup()| to the merged \emph{intracommunicator}, which makes the time consumption even larger. We expect a better method to handle this process so working nodes don't have to wait.
	\item When establishing communication between nodes, the current implementation uses an \emph{existing or newest} strategy (described in the \tIncDep chapter) to select MPI \emph{communicator}s. This behaviour is introduced because of dynamic spawning and is subject to change if point 3 is changed.
\end{enumerate}

\subsection{Possible Extension}
We consider our work an important step towards further extension or optimization of \dpy. Here, we present some of the possible further extensions or optimizations based on our extension. \\

\textbf{Dynamic scheduling}\quad
When executing workflows, it is always better to deploy PEs smartly to reduce the time consumed for data transmission or to deploy more computation-intensive PEs to powerful nodes and less computation-intensive PEs to less powerful nodes. Therefore, it is useful to be able to deploy PEs to specific nodes according to certain factors (\eg performance data), but this is an NP-hard problem. There are some work (\eg \cite{teylo2017hybrid}) focusing on static scheduling which use prior knowledge or performance data from previous runs; however, building upon our work, future development can gather not only this kind of data but also data from the current run. Moreover, if the deployment of \tPEInst{}s can be moved from one node to another, the runtime performance data will be more useful. \\

\textbf{Auto bundling}\quad
The workload of different PEs are not balanced in nature, meaning some PEs may have heavy computational jobs while some others may have very light ones. By bundling some light PEs together, the network traffic will become local data transmission so throughput will increase because local data transmission / sharing is significantly faster than network traffic. Moreover, less nodes are required so we may be able to utilize these free nodes to run more computation-intensive PEs (probably in the sense of \tPEDup{}s) to achieve better performance. \\

\textbf{Regionalized coordinator}\quad
We use a single coordinator in our implementation, which may consequently become a bottleneck (either to communication or computation). Some properties of the specific workflow graph may be useful to ease this constraint: we may analyse the workflow graph and identify separate regions (\eg connected by one PE) and assign each region one coordinator so each coordinator is only responsible for its region. This may be especially useful when the communication between worker nodes and the coordinator is expensive. \\

\textbf{Dynamic construction}\quad
In the current \dpy semantics, PEs are constructed before the execution of the workflow. It violates the semantics if the system modifies the PEs, and forces us to design \tdynexp as it is now. However, if we can construct PEs or at least modify some of its properties (in an approved explicit way) during runtime, we can make the dynamics of \tdynexp more robust - by changing the parameters used to construct the divisional PE to change its behaviour (from the coordinator side), rather than by message passing between PEs. This may be especially useful if we want to apply the existing grouping mechanism to expandable PEs.
