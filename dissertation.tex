%%%%
%% Load the class. Put any options that you want here (see the documentation
%% for the list of options). The following are samples for each type of
%% thesis:
%%
%% Note: you can also specify any of the following options:
%%  logo: put a University of Edinburgh logo onto the title page
%%  frontabs: put the abstract onto the title page
%%  deptreport: produce a title page that fits into a Computer Science
%%      departmental cover [not sure if this actually works]
%%  singlespacing, fullspacing, doublespacing: choose line spacing
%%  oneside, twoside: specify a one-sided or two-sided thesis
%%  10pt, 11pt, 12pt: choose a font size
%%  centrechapter, leftchapter, rightchapter: alignment of chapter headings
%%  sansheadings, normalheadings: headings and captions in sans-serif
%%      (default) or in the same font as the rest of the thesis
%%  [no]listsintoc: put list of figures/tables in table of contents (default:
%%      not)
%%  romanprepages, plainprepages: number the preliminary pages with Roman
%%      numerals (default) or consecutively with the rest of the thesis
%%  parskip: don't indent paragraphs, put a blank line between instead
%%  abbrevs: define a list of useful abbreviations (see documentation)
%%  draft: produce a single-spaced, double-sided thesis with narrow margins

\documentclass[msc,cs,deptreport,logo,abbrevs,11pt]{infthesis}

\usepackage[numbers]{natbib}
\usepackage{hyperref}
\usepackage{url}
\usepackage{listings}
\usepackage{amsmath}
\usepackage{graphicx}
\usepackage[normalem]{ulem}
\usepackage[skins]{tcolorbox}
\usepackage{xspace}\usepackage[inline]{enumitem}
\usepackage{etoolbox}
\usepackage{multirow}

\lstset
{ %Formatting for code in appendix
    language=Python,
    basicstyle=\footnotesize,
    numbers=left,
    stepnumber=1,
    showstringspaces=false,
    tabsize=1,
    breaklines=true,
    breakatwhitespace=false,
}

\newcommand{\rcpt}[1]{\emph{#1}}

\newcommand{\dpy}{dispel4py\xspace}
\newcommand{\Dpy}{Dispel4py\xspace}

\newcommand{\THE}[1]{the #1}

\newcommand{\tesieve}{sieve of Eratosthenes\xspace}
\newcommand{\teSieve}{Sieve of Eratosthenes\xspace}
\newcommand{\ttesieve}{\THE{\tesieve}}
\newcommand{\tsieve}{\tesieve}
\newcommand{\tSieve}{\teSieve}
\newcommand{\ttsieve}{\THE{\tsieve}}
\newcommand{\ttSieve}{\THE{\tSieve}}

\newcommand{\defTerm}[2]{%
  \expandafter\newcommand\csname t#1\endcsname{\emph{#2}\xspace}}

\defTerm{IncDep}{Incremental Deployment}
\defTerm{Incdep}{Incremental deployment}
\defTerm{incdep}{incremental deployment}

\defTerm{DynExp}{Dynamic Expansion}
\defTerm{Dynexp}{Dynamic expansion}
\defTerm{dynexp}{dynamic expansion}

\title{Making \dpy dynamic}
\author{Rui Zhao s1623641}

%% If the year of submission is not the current year, uncomment this line and 
%% specify it here:
% \submityear{1785}

%% Optionally, specify the graduation month and year:
% \graduationdate{February 1786}

%% Specify the abstract here.
\abstract{%
Although workflows and relevant technologies are being used widely, there are still some aspects that current workflow management systems (WMSs) did not cover. The data-streaming style is the future trend but current systems did not fully utilize the properties of data-streaming. We present two extensions, \tincdep and \tdynexp, to \dpy, a data-streaming WMS, to add more dynamics during runtime and enable future extensions and optimizations: \tincdep allows \dpy to incrementally deploy PEs to nodes during runtime rather than at the beginning of the execution; \tdynexp enables \dpy to dynamically manipulate the workflow according to actual needs. We build these two extensions on top of the MPI mapping of \dpy and demonstrate they two extend \dpy at a low cost and scale reasonably. We also show there is space to optimize so the overhead may reduce and these two extensions opens several windows for future extensions and optimizations.
}

\begin{document}
\begin{preliminary}
\maketitle

\begin{acknowledgements}
Many thanks to my supervisor, Prof.\ Malcolm Atkinson, who gave me many initial ideas, helped to improve the assumptions and designs and kindly assisted my write-up for the dissertation; and thanks to Rosa Filgueira, my co-supervisor, who kindly gave me many suggestions about the details and the dissertation thesis. Of course, thanks for the help and teaching of all my previous lecturers and teachers.

I also appreciate the previous work of the researchers in workflow-related technologies for they have enabled this research field and presented many valuable ideas and examples.

Also thanks to the help from many people online who helped me to figure out the usage of some toolkits, as well as the authors of many toolkits, including mpi4py which is one of the basis of our work.
\end{acknowledgements}

\standarddeclaration

%% Finally, a dedication (this is optional -- uncomment the following line if
%% you want one).
% \dedication{To my mummy.}

\tableofcontents

%% If you want a list of figures or tables, uncomment the appropriate line(s)
% \listoffigures
% \listoftables

\end{preliminary}


%%%%%%%%
\chapter{Introduction}
We report the exploration of improvements to a workflow management system that enables scientific methods to be encoded using data streaming. We report and evaluate a prototyping dynamic mapping of processes onto computational platforms.

The prevalence of workflow systems as a means of ongoing computational and data driven methods motivates this work. The nature of these systems and their data-streaming variants are introduced to set the context for the experiments. We then introduce \dpy as well as possible enhancements to it, and briefly describe our design to tackle them.

Science and researchers have gone through many eras and stages, and are facing a new paradigm today -- data-intensive science \cite{hey2009fourth}. Many current research campaigns involve producing large volumes of data and analysing them. As always, researchers need to develop methods themselves and then perform computation and analyse the results. Consequently researchers need flexible tools to help organise, implement and perform their methods. Here is how workflows and workflow management systems join the story: researchers can describe their methods by means of workflows and execute them in a workflow management system.

The scientific workflow (``workflow'' for short) is a technique used to organize and manage scientific computational jobs. By using scientific workflows, each computational job is decomposed into several (sub-)tasks which are connected by their dataflows (data dependencies). Hence, every workflow can be easily re-designed to suit specific needs by modifying its (sub-)tasks. Moreover, each (sub-)task can be seen as a module which can be developed independently, and they can be reused in different workflows so researchers can easily design new workflows by selecting and combining different modules.

The software system / framework used to execute scientific workflows is called a workflow management system (WMS). Researchers only need to provide the workflow and relevant input data, then the rest of the story will be automated by the WMS. Usually WMSs will parse the given workflow and map / assign tasks to some kinds of distributed computation platforms, such as clusters.

The \dpy system is a data-streaming workflow management system written in Python \cite{doi:10.1177/1094342016649766}. It does not introduce new execution platforms, but maps the workflow into some existing platforms such as MPI \cite{MPI_forum} or Apache Storm \cite{apache_storm}. As Python is widely available, the computers / servers on which the execution will happen don't need extra configurations. That said, as long as the libraries that tasks / PEs used are installed, \dpy can take control of the whole executing process and use the platform existing with no require of users' attention. If the tasks / PEs are written completely in Python, then no configuration is needed at all.

Currently, as far as we know, the workflow going to execute should be constructed solidly before its execution and is not subjected to change during its execution in most WMSs (which is all of the \emph{data-streaming} WMSs and most of the \emph{task-oriented} WMSs \footnote{The differenc between \emph{data-streaming} WMSs and \emph{task-oriented} WMSs will be detailed in the ``Background'' chapter.} with only a few exceptions, such as DAGMan for Condor \cite{couvares2007workflow}) and the deployment (\ie which node will execute which sub-task) happens at the beginning of the execution, which means the workflow as well as its deployment will be static during the execution. However, some computations can be represented more naturally in a dynamic way and some nodes (especially the ones near the end of the workflow) will actually be waiting for a long time before data coming in. As a consequence, it may save time or resources not to deploy all nodes in the beginning and may save nodes (therefore lower computational resources needed) if we can collect and re-allocate nodes that have finished producing outputs. To tackle this weakness, we extended \dpy to give it the ability to dynamically / incrementally deploy (sub-)tasks to computer nodes and a new semantics, \emph{dynamic expansion}, to workflows. In the meantime, our new system keeps backward compatibility so that existing workflows can still be correctly executed.

The ability to deploy (sub-)tasks to computer nodes only when needed is called \emph{incremental deployment} by us. We introduce a coordinator node that knows the global structure of the workflow and is in charge of node deployment, and degrade the rest of the nodes to executors / workers that only know themselves and their neighbours (\ie the nodes its input and output connections are connected with). Every time an executor needs new output targets, it requests that node and gets response from the coordinator. After all executors have finished their work (\ie no more inputs and outputs), they will all be shut down by the coordinator (and the coordinator also shuts itself down), so the whole execution is then finished. We use carefully designed signals / communications (which will be detailed in the \rcpt{Incremental Deployment} chapter) to ensure the execution and shut-down process's correct execution. This master-slave-like structure acts as a demonstration of possibilities for \dpy or other similar systems and is subject to change (\eg multiple coordinators \footnote{More details are presented in the \rcpt{Future Work} chapter.}) for future development.

The new semantics to grow the workflow dynamically according to certain rules is called \emph{dynamic expansion}. In order to keep backward compatibility, we add a special mark (implemented as a \lstinline|property| with default value) to the Processing Element (PE) indicating whether or not the task can be further expanded. In order to encode the expansion rules, we add special connections (called \emph{circuit connection}) and corresponding methods to define them. The details will be described in the ``Dynamic Expansion'' chapter.

Introducing these two new extensions enables us (or other researchers / developers) to make further optimisation and extensions. For example, in the original \dpy system, we can only collect performance data of past runs and use them to direct the deployment of future runs; but now, we can also collect the performance data during the incremental deployment (\ie on-the-fly) and use these data to aid the following deployment of the same execution.

The remaining part of this document will be organised as follows: first, we will present some terminology which may lead to confusion if not stated clearly; following that, we will present necessary background information; then we will detail the extensions we have done in separate chapters; after that, we will present the measurement and evaluation for them; finally, we will draw a conclusion.

	
\chapter{Terminology}
To avoid confusion, we present how we use terminology in this document.

\newenvironment{term_box}{\begin{tcolorbox}[enhanced,width=5in,size=fbox,
    fontupper=\large\bfseries,drop shadow southwest,sharp corners]}{
\end{tcolorbox}}

\begin{itemize}

\item \emph{PE} is short for Processing Element, which is the basic component of \dpy graph (detailed later). In most cases, a \emph{PE} is equivalent to a \emph{task} in a workflow. We will use the term \emph{PE} instead of \emph{task} when talking about \dpy.

\item \emph{Task} means a specific computation job in a workflow. Each workflow consists of many \emph{tasks}. Sometimes, we use \emph{(sub-)task} instead of \emph{task} for clarity, but they mean the same. It will be used when we are not talking specifically about \dpy.

\begin{term_box}
Although \emph{PE} and \emph{task} are usually interchangeable in our context, there is a difference in the common case. Normally, \emph{task} is used in the task-oriented environment so they will exit after processing; \emph{PE} is a term in \dpy (which is a \emph{data-streaming} system), so they will keep running and waiting for more data after processing a unit of data until no more data coming in.
\end{term_box}	
	
\item \emph{Node} means a computer in a network, usually one executing some computational jobs. Because modern computers usually consists of many \emph{core}s, it is not ideal to consider each of the computers (which may have different number of \emph{cores}) as a standalone entity. A better practice would be considering each \emph{node} as a single-core machine.

\item \emph{Core} refers to a CPU core. A modern computer consists of many \emph{core}s, so \emph{core} is not the same as \emph{node}. Usually we won't use \emph{core} directly because programming doesn't directly operate on \emph{core}s.

\begin{term_box}
Although \emph{node} and \emph{core} are different in definition, we would usually consider each \emph{node} as a single-\emph{core} machine, which means they are the same in our context (and we prefer to use \emph{node}).
\end{term_box}	

\begin{term_box}
Notice that each \emph{task} or \emph{PE} is usually programmed with no aware of multi-cores. However, they can wield multi-threading to perform necessary light-weight (i.e. not computational-intensive) tasks (such as communication).
\end{term_box}
	
\item \emph{Unit} means a piece of data produced by some \emph{task}s / \emph{PE}s. Usually, each \emph{task} / \emph{PE} will consume or produce many \emph{unit}s of data with the same format (on each data stream).

\item \emph{Data-streaming} means the data produced is not going to be cached on disk (or other permanent storage devices), but to be directly sent (\ie streamed) to the corresponding receiver node(s).

\item \emph{Pipeline} refers to the property that the following nodes start to process data when the first \emph{unit} of data is supplied (\ie it does not wait until all data has been produced by upstream \emph{PE}s).

\end{itemize}



\chapter{Background}
Because scientific workflows decompose computational jobs into smaller (loose-coupling) pieces, tasks, and use dataflow to connect them, various strategies can be chosen to accomplish the execution (and management) of workflows. Different choices of strategies lead to different behaviours of the workflow management systems (WMSs) and make them suitable for different scenarios. In this chapter, we introduce the growing use of scientific workflows, providing three examples in different disciplines, present the classification of workflow management systems (WMSs) and introduce some basic concepts of the WMS on which our work is based, \dpy. During the journey, we also gradually show why we favour data-streaming WMSs and why we choose \dpy.

\section{Workflow on Nowadays}
The grow and expansion of scientifc workflows went through many years. During the past 10 years, workflows have been adopted by growing number of communities and have influenced almost all researchers \cite{ATKINSON2017216}.
% Because of the abstract definition of workflows, researchers can always produce new methods and tools to refresh current implementations or methods. Malcolm \etal \cite{ATKINSON2017216} presented four benefits of it:
%\begin{enumerate}
%	\item The encoded meaning can sustain during the evolvement of technologies because they are largely independent of implementation.
%	\item The platforms can be mapped to are unlimited as of the semantics.
%	\item Components are easily reusable and re-designable.
%	\item Inter-discipline sharing is easy.
%\end{enumerate}

Many diciplines use workflows to perform their computational jobs. For example, in a recent paper about a new framework for gravitational wave detection \cite{gwave}, \citeauthor{gwave} use the workflows provided by PyCBC \footnote{https://ligo-cbc.github.io/} to perform the actual gravitational wave detection job. In seismology, seismic ambient noise cross-correlation, which is an important job for preprocessing / denoising raw data gathered, is often encoded as workflows and executed by a WMS (e.g. a hybrid system, Asterism \cite{Asterism}, uses seismic ambient noise cross-correlation as their demonstration). A recent workflow management system called DALiuGE \cite{wu2017daliuge} is demonstrated, in production, capable to process vary large quantities of data and is going to be used, as a prototype, in a consortium of Square Kilometre Array (SKA).

Although great progress and plenty of successful work (including those described above, and others \eg \cite{berriman2007generating} \cite{berriman2010application} in astrophysics, \cite{aiche2015workflows} in biochemistry) have been made, many areas are still under exploration in the field of workflow-related tools. For example, in the past time, most WMSs adopt the task-oriented style which may touch the bottleneck with the growing volume of scientific data whereas data-streaming style WMSs can significantly solve this problem \cite{doi:10.1177/1094342016649766}. Although we have the disk storage space, as Kryder’s law \citep{Kryders_law} pointed the storage density can double every 14 months, the IO speed doesn't increase that fast so caching the data of every stage onto disk will eventually become a huge bottleneck for task-oriented WMSs, unless new storage technologies emerge.

Another field is the workflow-sharing technologies. Different systems have different techniques to set up workflows or tasks, which may involve programming languages and paradigms. Therefore, although systems like myExperiment \cite{de2008design} and CrowdLabs \cite{Mates2011} exist, it is still hard to share tasks and workflows in a domain or implementation independent way. The only work we are aware of is the work by \citeauthor{GARIJO2017271} \cite{GARIJO2017271} which describes the semantics and a system to share workflows and tasks in the sense of the actual task they describe. 

\section{Task-oriented and Data-streaming}
Because the term workflow doesn't describe any details or standards, different systems usually use different methods to construct, manage and execute workflows. These systems are called workflow management systems (WMSs). Because each of them is composed of different trade-offs in their design, different WMSs have different characteristics and can not be roughly unified.

We see the way how WMSs schedule tasks and transport data as their main difference. Therefore we divide WMSs into these two types:
\begin{enumerate}
	\item Task oriented
	\item Data streaming
\end{enumerate}

\textbf{Task-oriented} WMSs usually decompose the workflow into each task, and execute them separately in different phases / stages. The data produced by each task will usually be stored to disk and feed into the following task (and then may be deleted). Systems like Kepler \cite{ludascher2006scientific}, KNIME \cite{Berthold:2009:KKI:1656274.1656280}, Galaxy \cite{blankenberg2010galaxy} and Pegasus \cite{deelman2015pegasus} are all developed as task-oriented workflow management systems and they are the majority of WMSs. The benefit of task-oriented WMSs is they have the entire control of the workflow's execution and can schedule the deployment according to needs (\ie the user can easily stop the execution at some point); they can also provide better fault-tolerance because data of each stage are cached on disk. However, this property is also its weakness: splitting tasks into stages will force a latter task to wait until all its previous tasks has finished, so the time needed will be longer. Moreover, it won't be able to support a continuous infinite data source because the stage used to execute that source will be infinite.

\textbf{Data-streaming} WMSs, on the other hand, stream small units of data directly from the prior tasks to the following tasks, often in a pipeline fashion. It is significantly different to task-oriented WMSs which move entire data between stages, and each unit of data can be arbitrarily complex (or arbitrarily simple such as a filename) in data-streaming systems \cite{doi:10.1177/1094342016649766}. Therefore, it will naturally support infinite data sources and there is no stages so no time will be wasted in waiting for the entire dataset is produced. As a result, the total execution time will be lower and users can get preliminary (partial) data when the first unit of outputs come out of the last task(s). On the contrary, fault tolerance of data-streaming WMSs will be weaker because there is no default mechanism caching intermediate data (so if a node fails, the whole workflow will need to restart). Even though, this weakness can be alleviated a bit by introducing intermediate tasks which passes the data while persisting them.

Although there are already several systems in the field, they don't completely satisfy the needs of current or near-future researchers \cite{•}. For example, with the development of IoT devices \cite{•}, there is a growing number of continuous infinite streams, \eg generated from sensors, which can be make use of. We can imagine researchers using continuous infinite streams as sources in workflows to make, for example, preliminary transformations to raw data in the future. However, because many systems are task-oriented, they are locked down to finite data so are not able to support this fashion. Therefore, we can say that data-streaming is the future. Thus, we expect to give mor e dynamics to data-streaming WMSs to better satisfy not only today's but also tomorrow's needs of researchers.

\section{Dispel4py}
\Dpy is a data-streaming pipeline-fashion WMS. It it built on top of the design of the Dispel workflow specification language \cite{atkinson2012data} while aligns more to scientists \cite{doi:10.1177/1094342016649766}. The most significant benefit of \dpy is that it maps the execution of workflows automatically to other platforms (\eg MPI) so it requires no user attention as long as the libraries needed are properly installed. This characteristic makes \dpy a comprehensive system, which is lacked by many systems. Moreover, because of the wide availability of Python, the large number of high-quality libraries for Python, it is easy to write new workflows or PEs (see next paragraph) for \dpy, and the ability to interoperate with C/C++ codes makes it possible to wrap existing workflows to execute in \dpy. These make it very easy to migrate to or use \dpy, and  PEs written in pure Python are usually platform-independent so sharing PEs and workflows is also very easy in \dpy.

In \dpy, the basic component is called Processing Element (PE). Generally, each PE corresponds to a task in the workflow. Each PE has zero or more inputs, and zero or more outputs (but not likely to be both zero because it's useless). A PE itself doesn't know where its inputs are from and where its outputs are to, which decouples PEs.

A workflow in \dpy is constructed by connecting output connections from one PE and input connections from another PE.

An example (split-merge) of workflow construction is made in \dpy is shown in Listing \ref{lst:wf_example} (taken from the original \dpy paper \cite{doi:10.1177/1094342016649766}).

\begin{lstlisting}[frame=single,caption={Example code of workflow construction in \dpy},captionpos=b,label={lst:wf_example},language=Python]
from dispel4py.workflow_graph import WorkflowGraph

pe1 = WordNumber()
pe2 = CountWord()
pe3 = Average()
pe4 = Reduce()

graph = WorkflowGraph()
graph.connect(pe1, 'output1', pe2, 'input')
graph.connect(pe1, 'output2', pe3, 'input')
graph.connect(pe2, 'output', pe4, 'input1')
graph.connect(pe3, 'output', pe4, 'input2')
\end{lstlisting}

\defTerm{PETmpl}{PE template}
\defTerm{PEInst}{PE}
\defTerm{PEDup}{PE instance}

There is a conceptual difference between \tPETmpl{}s, \tPEInst{}s and \tPEDup{}s, which may worth mentioning: a \tPETmpl is the logic (template) for processing data, usually implemented as a Python class; a \tPEInst is an instance (after construction) of such a class (\tPETmpl); \tPEDup is used only when we are talking about the (deliberate) duplications (copies) of the same \tPEInst. A \tPETmpl can be used to construct many different \tPEInst{}s and they can all behave differently because of, for example, the different parameters used to call the constructor. By default, each \tPEInst only has one \tPEDup so we usually don't distinguish between them (and we will favour the term \tPEInst); however, when deliberate duplication happens (\eg for better throughput as suggested by \citeauthor{doi:10.1177/1094342016649766} \cite{doi:10.1177/1094342016649766}), we will talk about \tPEDup{}s for better clarity.

The execution of \dpy is done by mapping the workflow to an execution platform (\ie an DCI), such as MPI, Storm or Multiprocessing which are currently supported by \dpy. This mapping behaviour is the benefit of the abstraction that \dpy brings because developers for workflows don't need to know anything about the platform (\eg hardware or middleware) that the workflow is going to be executed. This behaviour also gives the developers of \dpy freedom to optimize or extend the actual working procedures of \dpy without needing to worry about backward compatibilities too much.

For example, one of the current ``problem'' in \dpy is that it maps the whole workflow graph at once in the beginning, which may consume much time. As the method how workflows are constructed presents, the workflow in \dpy is a directed acyclic graph (DAG). Therefore, we can perform topological sorting to the graph, and obtain the sources by picking up the nodes with zero in-degrees. Thus, it is possible that we can deploy nodes only when needed (\eg when there are data sending to it) so we don't have to allocate all the resources in the beginning. This is the basic point of our first modification - \tincdep. By using this way, out modification works a bit similar to task-oriented WMSs which also use topological sorting for this purpose.

But we don't stop here - we also make use of the dynamics that \tincdep provides to achieve other goals. That is: because we can deploy PEs incrementally, we can also perform other modification to out workflow graph incrementally according to some rules. This is how our second target, \tdynexp, is done: to dynamically expand the workflow according to needs.



\chapter{Use cases}
We present here two examples that are used to explore the semantics and performance of \dpy. We require to retain all prior semantic interpretations of workflows but introduce new ones that depend on dynamic determination of the generated graph.

We will introduce them separately in the hope that it is more clear to read.

\section{\tSieve}
We first present \ttsieve which is a very simple yet useful algorithm. We present its basic idea, the common parallel algorithm for \ttsieve and the difficulty to construct it in the current semantics of \dpy (which is one of the motivations of our work).

The known earliest sieve algorithm is \ttesieve given by an ancient Greek mathematician \cite{o2009genuine}. The basic idea of \ttesieve is that we cross out all multiples (up until a certain upper boundary) of every prime at the time we encounter a new prime.

As we are using prime sieves as demonstration, we mainly focus on the correctness and simplicity rather than the efficiency. Therefore, we can simply change the structure of \ttesieve into a distributed manner: each node is responsible for one prime, and it crosses out all multiples of the prime it is responsible for; a continuous integer producer produces integers to the first sieve; each sieve node sends the un-crossed integers to the next sieve node.

As the description of the distributed sieve shows, we need each node responsible for each prime, which means we need to allocate at least number-of-prime sieves before executing the workflow. However, since we don't yet know which those primes are, we also don't know the number of them. This leads to a chicken-and-egg problem. One possible practice is to estimate the number of primes in the region and allocate that many nodes.

\newcommand{\cdIntGen}{\lstinline|IntegerGenerator|\xspace}
\newcommand{\cdSieve}{\lstinline|PrimeSieve|\xspace}

In the current \dpy system, to construct a workflow for this, we first define two kinds of PEs:
\begin{enumerate*}
	\item \cdIntGen which continuously produces integers from 2 up to a certain limit;
	\item \cdSieve which keeps a prime number it is responsible for and passes all number which is not a multiple of that prime. Then we need to connect the output of \cdIntGen to the input of \cdSieve , and then chain as many \cdSieve{}s as we need.
\end{enumerate*}

To define the graph, we need exactly two numbers: one is the range (\ie the maximum number) and the other is the number of primes in this region. Typically, we need to first find out the number of primes in this range by running a prime generator elsewhere, and then use it to construct the workflow. If we set the number smaller, the workflow can execute, but will produce unreliable numbers when the actual number of primes exceeds the number of sieves defined in the workflow graph; if we set the number larger, the correctness of the workflow execution will then depend on how outputs from the sieves are connected to the successor nodes (e.g. if only the final sieve is connected to the successor nodes, then there will be no outputs from the last sieve and, therefore, no inputs to the successor so it will behave erroneously).

Two problems emerge from this construction method:
\begin{enumerate}
	\item The chicken-and-egg problem previously mentioned;
	\item A different graph is needed when we want to change the maximum number.
\end{enumerate}

Both of them are quite unsatisfying for researchers / developers because they both involve manual inspection apart from the basic workflow (especially task) design. Researchers, \eg us, would prefer a more unified automatic way in which we only need to define once without calculating the number of primes before execution.

This motivates our research and we present the new semantics called \emph{dynamic expansion} (in addition to \emph{incremental deployment}) to support this expectation. In the new semantics, we no longer need to manually assign many sieves. Instead, we only need to define the \cdSieve automatically expandable (by setting the \lstinline|repeatable| property to \lstinline|True|) and describe when to expand it (by connecting \lstinline|circuit|). Moreover, in our new system, another new use case is available: to find certain number of primes - by making the \cdSieve nodes aware of their status and no longer expand more sieves (it will be more efficient if \emph{backward shut-down propagation} is implemented). Details will be described in the Dynamic Expansion chapter. 

\section{Seismic cross-correlation}

	
\chapter{Incremental deployment}
\sout{The hypothesis and assumptions of dynamic deployment. \\
The design of the system in order to support dynamic deployment. \\
The details how we suit the design to the existing dispel4py framework.}

As said in the Background chapter, we expect future researchers would need to design workflows reading data from continuous infinite data sources. Therefore, the execution time of the workflow will also be infinite. Because of the variety of input data, we expect a dynamic way of optimising the target where a task should be deployed. To achieve this, we need first introduce a way to dynamically deploy tasks to computational nodes, and only then can make decisions on-the-fly.

Therefore, we introduce \emph{incremental deployment} to accomplish the job to deploy tasks dynamically. The following part of this chapter will first present how we bootstrap our design of the system; then give a high-order description of how we design the system to support \emph{incremental deployment}; later describe in depth how we organise and build each part; finally present how we change our modification to keep the compatibility of existing dispel4py mechanisms.

\section{Bootstraping the Design}
To deploy nodes incrementally, we will need a mechanism to guarantee:
\begin{enumerate}
	\item All PEs who needs a target can get a target.
	\item All PEs who are connected to the same PE (target) in the workflow will get the same target when we dynamically deploy PEs. That can be decomposed into two parts:
	\begin{enumerate}
		\item Two PEs requiring  the same target simultaneously will always get the same target.
		\item Two PEs requiring the same target at the same time will only trigger one deployment action.
	\end{enumerate}
	\item There is no lost of data during execution.
\end{enumerate}

Because of the nature of distributed computing, we couldn't find a low-cost way (\ie without many communications) to guarantees all these requirements. Therefore, we decide to introduce a coordinator whose main job it to handle the assignment / deployment and we will keep the transmission / communications minimal and keep the workload of the coordinator minimal to in case it becomes the bottleneck.

Basically, when a PE wants to send data through an output connection to a target, it will need to know the address of the target PE. If it doesn't know that address yet, it will need to request for the address. This could be done in a pair of communications - the first is from the PE to the coordinator (requiring the target), and the second is from the coordinator to the PE (informing the address of the target). If the target doesn't exist at the time of the request, the coordinator postpones the reply and tries to deploy the target first (and then sends the reply).

Naturally, when multiple PEs are requiring a same target at the same time, the coordinator could sequentialize these requests so only the first request actually triggers the deployment action while others simply wait until the deployment succeeds. There is also a possible scenario that one output connection connects to multiple targets. The above strategy can be easily extended to handle it: just check all possible targets (instead of assuming only one target there is) to see if another request is dealing with it. This can be done in either a sequential way or in parallel.

To reduce transmissions and save time, when a PE receives the target of an output connection, it caches the targets locally. When it wants to send data through this output connection again, it can directly use the cached information and there is no need to communicate to the coordinator again because, at our current stage, the deployed PEs won't move and all targets are deployed already when the coordinator sends the reply. Therefore, we can finalize the strategy to send data through a connection:

FLOW CHART HERE

In the above discussion, we assumes the strategy of deployment is known. However, we haven't really discussed it yet. Now is the time to design it.

To deploy a PE to a node, the coordinator sends a ``deploy'' communication with the PE that is going to be deployed to the node. We can, as provided by python, give all nodes the knowledge of all possible PEs. Therefore, the communication doesn't need to contain the whole object - only the name and / or necessary instructions to construct the PE can give the target node enough knowledge to construct the PE on its side. After constructing the target PE, the target node tells the coordinator that it is ready to receive data and then the coordinator can do other stuffs (\eg reply the address of this target when needed).

Apparently, a workflow will usually contain many PEs, so the coordinator will need to store the information that which PE is deployed to which node. When further requests come, the coordinator will first lookup to see if the PE is deployed or not.

Data are streamed directly from PEs (nodes) to PEs (nodes), and they won't pass through the coordinator.

The last question is how to determine when the execution of the workflow has finished (if the source is finite). We follow the existing design of the system to use a propagation fashion. By appending an ``end-of-stream'' marker to the end (\ie last unit) of each data steam, a PE can know how many previous nodes have finished producing outputs. Apparently, we will only need to give each PE the knowledge of the total number of previous nodes and then the PE can tell if all previous nodes have finished or not. If all previous nodes have finished producing outputs, the current node knows there won't be more inputs and therefore no more outputs will be produced by itself, so it can send (propagate) the ``end-of-stream'' marker to all its following PEs. Although it seems that we can do this on either PE level (when a PE has finished producing more data) or connection level (when no more data is going to be sent through a connection), in reality we prefer to do it on PE level because determining whether no more data is going to be sent through a connection requires the programmer of the PE to create extra logic to indicate this in a pipeline system.

\section{General design}	
As described above, we introduce a coordinator in addition to the PEs defined in the workflow, and degrades other nodes as executors (while still maintains the existing ``wrapper'' mechanism to PEs). The structure of our system is:

HERE SHOULD BE A FIGURE

The coordinator possesses the knowledge of the whole graph and current assignments, and handles deployment. Deployment happens when a node (which has been deployed previously) sends a ``require'' message along with the name of the output connection. When receiving this message, the coordinator first checks whether the target PE is already deployed or not: if it is already deployed, the coordinator simply replies it; if not, find nodes that are suitable to deploy, send deployment signal and reply with the assignment. Finally, when all nodes are finished (\ie no node is working and no deployment on-the-way), the coordinator also sends the ``finalize'' signal to all executors and then the whole execution of the workflow is shut down.
 
The main job of an executor is to execute one PE (in a wrapper) in the workflow. Apart from executing the PE (inside the wrapper), it also receives the deployment of itself (from coordinator) and receives shut-down (``finalize'') messages. All other actions are done in the wrapper.

A wrapper is the place that handles data receiving and sending of a PE. In addition to these original functions, we extend the wrapper to suit the need of incremental deployment - to request targets when an output is going to be sent through a new output connection, and wait for coordinator's reply. Specifically, an ``end-of-stream'' marker is sent after the last unit of data through all related output connections and the wrapper shuts itself down and returns control to the executor; when a wrapper receives enough (\ie the same as the sum of the number of nodes from each input connection) ``end-of-stream'' markers from its previous nodes, it then knows there is no more data and then propagates the ``end-of-stream'' marker through all its output connections the same way as previously described; in addition to propagation of the ``end-of-stream'' marker through the workflow graph, when each wrapper shuts itself down, it also sends a ``terminated'' signal to the coordinator so the coordinator knows this node is free (\ie back to the initial ``executor'' state and can then reassign it another PE if any.

Finally, there is an existing ``grouping'' mechanism which will create some duplicated PEs when deploying, and scatter data among them all so each of them could have fewer data to process and may reduce the total execution time. Our system introduces a local leader (called ``representative'') in each group who is in charge of the synchronization between groups. The detailed synchronisation strategy will be discussed below.

\section{Structure of Components}
After finishing the general design of the system, we will now suit the design to our actual platform. We will consider these points: efficiency and concurrency, synchronization, compatibility (of existing designs), and spawning (more node when needed).

\subsection{Efficiency and Concurrency}
We described a bit in the above  about the possibility of concurrency in the components. Making them run concurrently can usually bring better efficiency.

Because each PE is self-contained, we use multi-threading to achieve concurrency. It should be mentioned that each PE is designed to be single-threaded (at least, behaves like single-threaded).

The first is the coordinator. We can expect many communication to and from the coordinator run in parallel. Therefore, we will handle each request in a separate thread. Because multiple requests may act on a same PE, we will add a lock to each PE and acquire it when it is going to be act on (and release it when finished) so different requests of this same PE will be sequentialized.

Then is the executor. Because the function of executors is very simple and will only receive data, there is no need to make them run in multiple threads. Therefore, we design the executor to run in a single-threaded way.

Finally is the wrapper. Because a wrapper reads data from some input connections, processes the data previously read and writes processed output to some output connections, we run ``read'' in one thread, and run ``process'' and ``write'' together in another thread. If the order of data doesn't matter, we will run both ``process'' and ``write'' in several threads (in a thread pool). Because we need also request the address of targets, we need to design a suitable place to handle the communication to the coordinator. We decide to send the request at where is needed, and listen in another dedicated thread. The synchronous will be done by using condition variables - wait for the condition variable to become true after sending the request, and set it to true when receiving the reply.

\subsection{Synchronization}
One important aspect of distributed computing is to make sure the parallel execution won't break the needs of ordered messages \cite{•}. Though not all messages need to be strictly ordered, some important messages should be kept ordered. Therefore, synchronization between nodes is very important.

In our system, there are several causalities which imply message ordering: receiving the target happens after target is deployed; ``end-of-stream'' marker sends and receives after all data units (which also implies shutting down only after receiving all data units). It becomes even more complicated when we try to handle the ``grouping'' mechanism: not all nodes inside a group will send data to some targets, so how to make sure shut-down propagation works correctly?

Luckily, MPI guarantees ordering of messages between each pair of nodes in one communicator \cite{MPI-3.0}, data won't be lost until it is received or cancelled and it provides several synchronization options for receiving and sending data. This slightly reduces our workload: the coordinator can always safely reply targets to the PE regardless of whether the target has finished initialization or not because the data won't be lost (so we don't need to do synchronization to make sure the target is ready); if we send the ``end-of-stream'' marker after all data units, the receiver side will not receive it before any data units.

However, we still need to carefully design the system especially when we try to keep the ``grouping'' mechanism. By introducing the local leader (\ie representative), we will use it to do synchronization of shut-down propagation from group to group, and all other nodes in that group (called ``brothers'' to the ``representative'') will not do synchronization across groups. Basically, we are aware that it is impossible for an unknown number of nodes (without doing explicit synchronization) to correctly propagate ``end-of-stream'' marker - because the number of sources is unknown to the receiver side. Therefore, we make the representative in charge of propagating ``end-of-stream'' marker. Roughly, the behaviour could be drawn like this:
\begin{enumerate}
	\item The ``end-of-stream'' marker will send to all groups and all nodes in each group.
	\item All nodes in the group who has finished producing more data will tell the representative that they are going to shut down and then they can shut down.
	\item Upon receiving a shut-down message from inside the group, the representative will check if that's the last node (including itself) who is going to shut down. If it is, then the representative propagates the ``end-of-stream'' marker to all following nodes and then shut itself down.
\end{enumerate}

This strategy can solve the problem of unknown number, but will introduce another problem: the order of messages is only guaranteed for each pair of nodes, but not across pairs of nodes. That means the ``end-of-stream'' marker sent by the representative (node A) may be received (by node R) prior to the last unit of data from one of the brothers (node B), where A and B are in one group.
 Therefore, we need a way to determine whether the receiver side has received all data from the previous group and then the representative can safely propagates the ``end-of-stream'' marker. With the combination of the 
\lstinline|MPI_Issend()| function and the \lstinline|MPI_Waitall()| function of MPI, we can make each brother wait until all the data sent from it have been received by the receiver side; then, the brother can tell the representative that it is going to shut itself down. Thus, when the representative knows all brothers are going to shut down, all data has already been received so propagating the ``end-of-stream'' marker at that time will not violate any constraints.

\subsection{Compatibility}
Compatibility here refers to both try to support all existing features and try to keep the existing structure of the design, because the current design of dispel4py is used in all parts (all mappings) so if we change one part of the core design, the rest of the code base usually should also be changed.

We have discussed some of the compatibility issues in the above part, such as how we keep the existing ``grouping'' feature while adding incremental deployment. In addition to that, we also slightly modified the design of PE to add an extra field (property) - FIFO. By defining the PE to be FIFO, the wrapper then knows it should not run multiple \lstinline|pe.process()| concurrently, but should execute them sequentially so the outputs are kept to be FIFO. Another important aspect that we haven't discussed yet is the \lstinline|Communication| class used in the dispel4py framework to decide which  target node should the current unit of data be streamed to. A \lstinline|Communication| is constructed with the addresses of target nodes and the connection type. When deciding the target of a unit of data, the name of the output connection is used. Therefore, we will construct the \lstinline|Communication| of an output connection right after the reply of target addresses.

\subsection{Spawning}
Traditionally, when running an MPI programme, the user needs to decide the number of nodes that is going to be used at the time of execution. However, this method has two drawbacks:
\begin{enumerate}
	\item Spawning nodes takes time, and the more nodes you spawn the more time it needs.
	\item Some PEs may terminate earlier than others so actually the maximum number of nodes running simultaneously may be smaller than the number of PEs in the workflow (so some nodes will be spawned but never used).
\end{enumerate}

Therefore, we expect to spawn nodes only when needed during running. MPI provides this mechanism by a function \lstinline|MPI_Comm_spawn()|. There is a configurable parameter which controls the number of nodes going to spawn at this time. The spawned nodes will be placed on a different \textit{intracommunicator} and connect with the existing nodes through an \textit{intercommunicator}. Because we want all executors to be able to communicate with each other and we may do the spawning for multiple times, we will also have to merge the \textit{intercommunicator} to a new \textit{intracommunicator} by using \lstinline|MPI_Intercomm_merge()|.

Then, we will need to design the strategy of communication under the scenario of spawning and merging. The strategy can be chosen between these two ends:
\begin{enumerate}
	\item Switch from old communicator to new communicator every time the new a communicator is created. 
	\item Use the old communicator whenever possible. 
\end{enumerate}

Our strategy choice is somewhat between these two: each communication channel always goes through the same \textit{communicator}; each every communication always uses the newest \textit{communicator}. 

	
\chapter{Dynamic Expansion}
In this chapter, we describe further why we introduce \tdynexp, and present how we design this feature. After that, we describe how to construct \tsieve in \tdynexp and finally present the problem we encounter during implementing \tdynexp.

We described \tsieve in the Use Cases chapter and it is our main motivation to allow developers to construct workflows like that more naturally and universally. By introducing \tdynexp, we expect to allow developers to design \tPETmpl{}s which are elastic according to actual needs while keeping backward compatibilities with the existing \dpy mechanism.

\section{PE Definition}
To introduce the dynamics, we need to add some extra mechanisms to the definition of PEs (\ie \tPETmpl{}s). We now present the general need, followed by our implementation.

In general, each PE should decide whether to \textit{expand} or not when receiving and processing data. We use the term \textit{expand} to limit this dynamics to a handleable range or the range will explode. By saying \textit{expand}, we expect each PE behave like cell division, to split to another almost the same PE, when the mechanism is triggered, and we limit this mechanism to trigger at most once for each expandable condition each PE (and the divisional PE will be treated as if it is a PE defined in the workflow so it can still \textit{expand}).

Although we limit this mechanism to \textit{expand} for our prototype system, it is capable to accomplish the design of \ttsieve as well as other workflows containing a similar behaviour. Moreover, by combining the expansion behaviour with the mechanism used to bundle multiple PEs together to form a large PE (which is an unnamed mechanism in \dpy), developers can design more complex expansion behaviours.

In the above discussion, we deliberately blurred the difference between \tPETmpl and \tPEInst for simplicity, and we believe this won't cause confusion for readers.

To implement \tdynexp, we add two more features to the definition of each PE (\tPETmpl):

\defTerm{circuit}{circuit}

\begin{enumerate}
	\item A \textbf{property} used to inform the coordinator (and the developer) whether this PE (\tPETmpl) is expandable or not. \\
	This is implemented as a read-only property called \lstinline|repeatable| in the definition of PE, and defaults to \lstinline|False|. The developer only needs to override this property to \lstinline|True| when designing an expandable PE.
	\item A \textbf{mechanism} used to control when to trigger the expansion and how to communicate to the expanded PE. \\
	We add a special type of output connections called ``\tcircuit''. Each \tcircuit is a pair of connections (one input and one output), both defined in the same \tPETmpl. When the PE sends data through an output connection defined in the \tcircuit, the expansion mechanism is trigger, a division is deployed and the data is sent to the paired input connection in the \tcircuit. This \tcircuit then becomes a static connection, meaning this (old) PE can keep sending more data through this output connection and no more division will be triggered.
\end{enumerate}

\section{Runtime Behaviour}
The section above described how \tdynexp is designed in the PE part, and briefly mentioned the meaning of each of these features. This section describe how these features are expected to work during runtime, and then describe it in our implementation, mainly from the perspective of the coordinator.

An expandable PE (\emph{origin}) sends sends a signal containing the \tcircuit output connection. The system recognizes this is a request for expansion, prepares the divisional PE (which is almost a ``duplication'' to the PE (\emph{origin}), with extra configurations) and deploy the divisional PE so the \emph{origin} can subsequently communicate with the divisional PE.

Our implementation builds on top of \tincdep, and utilize many of its features. In our implementation, an expandable PE outputs data through a \tcircuit output connection using the same way as other output connections -- sending a signal to the coordinator. The coordinator checks if the PE(\emph{origin} is expandable, and then checks if the output connection is an output connection defined in \tcircuit. If both true, the coordinator will create a duplication to the PE (\emph{origin}), and deploy it to a node, using the same way as a common \tincdep.

To avoid confusion, when sending the ID of PE to a node during deployment, the divisional PE will contain a separate field with an increasing integer count. When the node receives this message, it extracts the PE ID part to obtain the PE, and ignores the other part (\ie the counter) in the current implementation.

\section{Simple Example}
In this section, we use \ttsieve as our example to show the difference of the semantics before and after implementing \tdynexp.

TO BE CONSTRUCTED

	
\chapter{Evaluation and Measurement}
In the previous chapters, we showed our modification to \dpy keeps backward compatibility and extends the semantics. In this chapter, we show the performance measurement based on several tasks and platforms to demonstrate the usability of our modification.

We demonstrate on two kinds of workflows: the prime sieve and the cross correlation, both have been discussed in the Use Cases chapter. The cross correlation workflow is from the previous \dpy paper and is largely computation-intensive. We expect to demonstrate the overhead of \tincdep is quite small compared to the actual computational job, and therefore can be neglected. The prime sieve was shown above (both in \ref{sec:uc_sieve} and \ref{sec:dynexp_example}) and is not computation-intensive. We use it to show \tdynexp works correctly and the benefit \tdynexp brings.

To show the performance, we describe here what platforms we choose to perform the measurement. Because this is an MSc project, except for my laptop, we use two shared-resource platforms to perform our measurement: the teaching cluster of the School of Informatics (InfCluster) and the university's cluster (EDDIE).
The configurations of the machines vary among all nodes in the clusters, but each cluster has some typical configurations. They are listed in Table \ref{tbl:list_measurement}.

\begin{table}[h]
\centering
\begin{tabular}{|c|c|c|c|}
\hline
 & Laptop & InfCluster & EDDIE \\ \hline
CPU & Intel Core i7 5500U & Intel Xeon E5 2650 (v3,v4) & Intel Xeon E5 2630 v3 \\ \hline
CPU (cores) & 4 & 32, 40, 48 & 16 \\ \hline
RAM (GB) & 8 & 64 & 64, 128 \\ \hline
\end{tabular}
\caption{Typical configurations of measurement platforms}
\label{tbl:list_measurement}
\end{table}

As shown in the table, we use several configurations on the clusters. The EDDIE cluster restrict users to use only multiples of 16 as the number of cores. It also restricts us to use at most the same number of cores the number of MPI processing elements, so we can't measure large sieves on it.

We use several different workflows and configurations to test different aspects of the system's performance. For the prime sieve, we use several different ranges, both the static version and the dynamic version, to test the correctness of our system. We also very the number of nodes and the number of initial nodes (for the dynamic version). For cross correlation, we vary the number of nodes to see how performance changes.

\section{Incremental deployment}
The working process of \tincdep was described in \ref{sec:incdep_example} and the correctness was also shown there. This section focuses on the performance issue.

As said above, we use the cross correlation workflow (xcorr) as the main workflow to demonstrate the performance. We run the workflows under different configurations, as describe above. Generally, for both workflows, we use one more node for the \tincdep version, which is required because of the mechanism of \tincdep and is an unavoidable overhead. One exception is when we reach the maximum number of cores requested on EDDIE: we only use as many as the number of cores for the number of nodes because of the restriction of the platform. To make it easier to read, we show only the number of worker nodes for \tincdep -- by subtracting $1$ from the number of nodes\footnote{However this will lead to unalignment when the number of processes is the same  as the number of cores on EDDIE.}. We present the results in Figure \ref{fig:xcorr_infcluster} and \ref{fig:xcorr_eddie}.

\begin{figure}[h]
\centering
    \includegraphics[width=1\textwidth]{figures/xcorr_infcluster_30day_2tr}
\caption{Execution time of XCorr (with plotting) on InfCluster with 2 traces in 30 days}
\label{fig:xcorr_infcluster}
\end{figure}

\begin{figure}[h]
\centering
    \includegraphics[width=1\textwidth]{figures/xcorr_eddie_90day_2tr}
\caption{Execution time of XCorr (with plotting) on EDDIE with 2 traces in 90 days}
\label{fig:xcorr_eddie_2tr}
\end{figure}

\begin{figure}[h]
\centering
    \includegraphics[width=1\textwidth]{figures/xcorr_eddie_90day_4tr}
\caption{Execution time of XCorr (with plotting) on EDDIE with 4 traces in 90 days}
\label{fig:xcorr_eddie}
\end{figure}

We explain here the notations in the figures:
\begin{itemize}
	\item The top of each figure is two of the configuration options: the number of traces and the duration of the data collected from.
	\item The dashed lines shows data / performance from the original version of \dpy; the solid lines shows data from our modifications of \dpy.
	\item The dots (whether they are points or crosses) represent individual performance data of the line with the same colour.
	\item The errorbar shows one standard deviation away from both sides of the mean.
	\item In the legends, there are several sections in the texts.
	\begin{enumerate}
		\item The first section is the system version we used; 
		\item The second section (wrapped in parentheses) shows the recording method of time and defaults to \textbf{shell} when omitted.
		\item The third section shows the platform and the number of cores requested o perform this measurement, and there may be some additional information (useful for us, but unlikely for the readers).
	\end{enumerate}
\end{itemize}

In the figures, we show two different recordings of time: one is inside the program (notated as \textbf{inside}) and the other is outside the program (noted as \textbf{shell}). The reason we use two recordings is that we want to also investigate the time consumed during programme running rather than the time for both initialization and running.

We have removed some significant outliers from the data used to draw the figures. There are still some data points outside of one standard deviation but we can not sensibly remove them because they are not apparently away from other data points. These outliers could be caused by the resources used by other processes in the OS of the execution machine(s) because we are using shared-resource platforms to perform our measurement.

As shown in figures, the total execution time is not far between the original \dpy version and our modification (\tincdep). The time difference is almost always constant (less than 10 seconds) so we have confidence that it won't increase when the total execution time increases and is acceptable when using under production -- because the total execution time would be significantly longer than that in our demonstration so this small difference won't even be noticed. The difference between each two time recordings of each configurations also shows that the time consumed to initialize one extra process (for the coordinator) isn't significant.

Notice in the second half of Figure \ref{fig:xcorr_eddie} when the number of processes is larger than needed, our version of system consumes almost constant time while the original version varies much. We define \emph{needed} as: processes are just enough to fully parallelize the computation. Because of the definition of the cross correlation workflow, the number of cross correlation is $C^2_n$ \footnote{Also denoted as $\binom{n}{k}$.} for $n$ traces where $C^k_n$ represents the number of $k$ combinations in $n$ elements. This leads to $1+C^2_n \cdot 2$ processes for $n$ traces when the workflow does not contain the plotting step and $1+C^2_n \cdot 2$ when the workflow contains the plotting step. For 2 traces, the number of needed processes is 3 (or 4 if including plotting); for 4 traces, the number of needed processes is 13 (or 19 if including plotting). We argue this is because the number of cores of each machine in EDDIE is 16 so 32 cores involves inter-machine communication so using \lstinline|comm.Barrier()| (which is what we added to the original version of \dpy to synchronize and record time) takes more time compared to our one-sided communication (only the coordinator sends signals to workers) in \tincdep. We believe this also forms the reason why the ever-increasing time happens in Figure \ref{fig:xcorr_infcluster} and \ref{fig:xcorr_eddie_2tr} because increasing the number of processes is pure overheads when there are only 2 traces.

Regardless of these phenomena, the time consumed between our new system with \tincdep and the original system without \tincdep is quite close, as stated above. Also notice there are some compromises in our implementation so there is still space to get better performance (\ie less time consumption). We believe this won't add significant overheads to a production environment so our extension is acceptable compared to the potential benefits and future optimizations it may bring.

\section{Dynamic expansion}
\tDynexp is the new property we have added to dispel4py. We use the prime sieve as our example to demonstrate how the system performs when using \tdynexp. The details of the workflows were described in \ref{sec:dynexp_example}.

The results are shown in Figure \ref{fig:sieve_opt_100} and \ref{fig:sieve_opt_1000}. The third section of the legends (in the figures) is one of the parameters controlling the number of processes spawned at the beginning.

\begin{figure}[h]
\centering
    \includegraphics[width=1\textwidth]{figures/sieve_opt1_100}
\caption{Execution time of the prime sieve to find prime up to 100 on both the old semantics and our new semantics}
\label{fig:sieve_opt_100}
\end{figure}

\begin{figure}[h]
\centering
    \includegraphics[width=1\textwidth]{figures/sieve_opt1_1000}
\caption{Execution time of the prime sieve to find prime up to 1000 on both the old semantics and our new semantics}
\label{fig:sieve_opt_1000}
\end{figure}

Similarly, we also removed significant outliers. The number of outliers is larger in InfCluster compared to EDDIE. This may be because the number of cores of each machine in InfCluster is larger than that of EDDIE and each machine can host more jobs, so more people are sharing the same machine. Another possible reason is because the prime sieve is not computation-intensive so the effect and latency of communication govern the performance but they are more unpredictable than the computation.

The x axis in these figures is different from the previous figures (\eg Figure \ref{fig:xcorr_eddie}) -- the ``number of iterations'' means how many times the producer repeatedly produces numbers from 2 to the upper bound. One reason is because we are no longer evaluating the performance difference caused by introducing \tincdep; the other reason is we want to show the consistency of the performance -- when increasing the number of iterations, more data (numbers) go through the workflow but no more PEs are deployed or created.

As we can see in the figures, there is no significant indication of whether the dynamic version is better or the static version is better. We argue that this is because much time is used to initialize for MPI communication. The dynamic version spawns significantly fewer nodes in the beginning so it can start early and adjust during runtime (so the leftmost points always show the dynamic version performs better). 

When increasing the number of iterations, the producer node executes longer so it can not be reused before finishing producing. Therefore, the coordinator can not reuse the producer node and also other nodes, so it is forced to spawn more nodes. This (and also producing more data takes more time) explains why the time consumed in execution increases when increasing the number of iterations.

We also tried several different sets of processes spawned in the beginning for \tdynexp. The result matches our intuition: the more processes spawned in the beginning, the more time consumed to initialize (by comparing points at 1 iteration); the more processes spawned in the beginning, the less likely spawning would happen during runtime (by comparing the number of sharp performance changes of each line).

The spawning process during runtime also takes time and may be optimised (see the \rcpt{Conclusion and Future Work} chapter) in order to get better performance for the dynamic semantics.

To draw a conclusion: in both figures, we can monitor better performance at certain points for the dynamic version. We believe they are because the dynamic version reused nodes so fewer nodes were spawned in total. This matches our expectation and may open a potential future optimization: how to identify what nodes can be reused according to the workflow.


\chapter{Conclusion and Future Work}
In this document, we presented the important role of scientific workflows for nowadays scientific researches, the classification to workflow management systems (WMSs) and \dpy, a data-streaming WMS, as the background information to our work. Then, we discussed two potential extensions, \tincdep and \tdynexp, to \dpy both will give \dpy more dynamics and enable further extensions or optimization and described our design and implementation of them. Finally, we presented our evaluation and measurement of our extension and showed that our extension can give \dpy more dynamics in a low cost and sometimes even with better performance.

Apparently, we haven't reached the best performance of the work because there are some imperfect aspects in the implementation due to the time limitation. There are also some potential further work could be done built upon our current work. Even though, we consider we have made good work in the time limit of two months.

\section{Future Work}
The rest part of this chapter will describe possible future work. They consist of two aspects:
\begin{enumerate}
	\item Possible optimization to our current implementation;
	\item Possible extension built upon our work.
\end{enumerate}

If we were allowed more time, we will explore the possible optimization first, and then the extension.

\subsection{Possible Optimization}
In our current implementation, there are some trade-offs (which have been briefly mentioned in the above chapters) because of the time limit. Given enough time, we will explore which way will be better and implement the best. We bring them together and describe a little further here:

\begin{enumerate}
	\item The \emph{process} and \emph{write} steps are combined together, and they are in parallel to the \emph{read} step during the processing of one \tPEInst. However, all them three can be executed in parallel in principle and, if needed, also satisfy the FIFO requirement by using coarse-grained scheduling or by using numbers.
	\item When spawning more nodes, the current implementation always spawn a fixed pre-defined number of nodes. However, there may be a smart way to specify this number dynamically according to the workloads.
	\item After spawning new nodes, the current implementation will merge all nodes together to one \emph{intracommunicator} in order to make all worker nodes inter-communicable. However, in MPI implementation, the merging operation takes time and during this time no data / message can be transmitted. We will call \lstinline|Comm.Dup()| to the merged \emph{intracommunicator}, which makes the time consumption even larger. We expect a better method to handle this process so working nodes don't have to wait.
	\item When establishing communication between nodes, the current implementation use an \emph{existing or newest} manner (described in the \tIncDep chapter) to select MPI \emph{communicator}s. This behaviour is introduced because of dynamic spawning and is subject to change if point 3 is changed.
\end{enumerate}

\subsection{Possible Extension}
We consider our work an important step towards further extension or optimization of \dpy. Here, we present some of the possible further extension or optimization based on our extension. \\

\textbf{Dynamic scheduling}\quad
When executing workflows, it is always better to try to deploy PEs to nodes nearby to reduce the time consumed for data transmission or to deploy more computation-intensive PEs to powerful nodes and less computation-intensive PEs to less powerful nodes. Therefore, it is useful to be able to deploy PEs to specific nodes according to performance data. Building upon our work, future development can gather data from both previous runs and current run. \\

\textbf{Auto bundling}\quad
The workload of different PEs are not balanced, meaning some PEs may have heavy computational jobs while some others may have very light ones. By bundling some light PEs together, the network traffic will become local data transmission so throughout will increase because local data transmission is always faster than network traffic. Moreover, less nodes are required so we may be able to utilize these free nodes to run more computation-intensive PEs to achieve better performance. \\

\textbf{Separated Coordinator}\quad
We use a single coordinator in our implementation, which may consequently become a bottleneck (either to communication or computation). We may analyse the workflow graph and identify separate regions (connected by one PE). Thus, we can assign each region one coordinator so each coordinator is only responsible for its region. This may be especially useful when the communication between worker nodes and the coordinator is expensive.



%%%%%%%%
%% Any appendices should go here. The appendix files should look just like the
%% chapter files.
%\appendix
%\include{appendix1}
%% ... etc...

%% Choose your favourite bibliography style here.
%\bibliographystyle{apalike}
\bibliographystyle{plainnat}

%% If you want the bibliography single-spaced (which is allowed), uncomment
%% the next line.
% \singlespace

%% Specify the bibliography file. Default is thesis.bib.
\bibliography{dissertation_bib,sieve}

%% ... that's all, folks!
\end{document}
